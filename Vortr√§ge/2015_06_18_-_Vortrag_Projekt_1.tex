\documentclass{beamer}

\usepackage[ngerman]{babel}
\usepackage[utf8x]{inputenc}
\usepackage{amsmath,amsfonts,amssymb}
\usepackage{tikz}
\usepackage{graphicx}

\usetikzlibrary{shapes,arrows,positioning,hobby,decorations.markings,fit,backgrounds,calc}

%\usetheme{Singapore}
\setbeamercovered{transparent}

\setbeamertemplate{section in toc}[sections numbered]
\setbeamertemplate{subsection in toc}[subsections numbered]
\setbeamertemplate{subsubsection in toc}[subsubsections numbered]

\newcommand\subsectnum{%
  \number\numexpr \insertpagenumber-\insertsubsectionstartpage+1\relax.~%
}

\begin{document}
\title{Projekt Straßenschilderkennung\\Präsentation Idee}   
\author{Artem Prokop\\
Theodor Malaki\\
Eike Florian Petersen} 
\date{\today} 

\frame{\titlepage} 

\frame{\frametitle{Inhaltsverzeichnis}\tableofcontents} 


\section {Vorstellung des Teams}
\frame {\frametitle{\thesection. \insertsection}
\begin{tabular}{l l}
\hline
\textbf{Name} & \textbf{Rolle} \\
\hline
Artem Prokop & Vorführung \\ 
Eike Florian Petersen & Vortrag \\
Theodor Malaki & Dokumentation \\
\end{tabular}
}


\section {Projektdefinition}
\frame {\frametitle{\thesection. \insertsection}
Der Aufbau der Bilderkennung soll anhand des Papers ``A Robust Algorith for Detection and Classification of Traffic Signs in Video Data'' von Tanh Bui-Minh, Ovidiu Ghita, Paul F.Whelan and Trang Hoang umgesetzt werden.
\newline
\newline
\newline
\begin{tabular}{l l}
Eingabeformat: & Einzelbild \\
Ausgabeformat: & Bildliche Darstellung der Funstelle/n \\
 & und textuelle Beschreibung der gefundenen Objekte \\
\end{tabular}
}

\section {Umsetzungsidee}
\frame {\frametitle{\thesection. \insertsection}
% define block styles
\tikzstyle{block} = [rectangle,fill=gray!10,draw,text width=10em,text centered,minimum height=3em]
\tikzstyle{cloud} = [ellipse, fill=gray!10,draw, text width=5em, text centered, minimum height=3em]
\tikzstyle{line} = [draw, -latex']

\resizebox{0.66\linewidth}{!}{%
  \begin{minipage}{\linewidth}
      \begin{tikzpicture}[->,>=stealth', auto, node distance=1cm, main node/.style={circle,fill=blue!15,draw,font=\sffamily\bfseries}]
      % place nodes
      \node [block, text width=13em] (1) {\centering Einzelbild lesen};
      \node [block, right of = 1, node distance=5.7cm, text width=13em] (2) {Bild vorbereiten (3.1)};
      \node [block, below of = 2, node distance=1.5cm, text width=13em] (3) {Schilderkennung \\ durch Form (3.2)};
      \node [block, below of = 3, node distance=1.5cm, text width=13em] (4) {Schildklassifikation (3.3)};
      \node [block, below of = 4, node distance=1.5cm, text width=13em] (5) {Schild entzerren (3.4)};
      \node [block, below of = 5, node distance=1.5cm, text width=13em] (6) {Schild erkennen (3.4)};
      \node [block, right of = 6, node distance=5.7cm, text width=13em] (7) {Einzelbild schreiben \\ textuelle Dokumentation erzeugen};
      
      \path [line] (1.east) -- (2.west);
      \path [line] (2) -- (3);
      \path [line] (3) -- (4);
      \path [line] (4) -- (5);
      \path [line] (5) -- (6);
      \path [line] (6.east) -- (7.west);
      
      \end{tikzpicture}
  \end{minipage}
  }
}


\subsection {Bild vorbereiten}
\frame {\frametitle{\thesection.\thesubsection~ \insertsubsection}
% define block styles
\tikzstyle{block} = [rectangle,fill=gray!10,draw,text width=10em,text centered,minimum height=3em]
\tikzstyle{cloud} = [ellipse, fill=gray!10,draw, text width=5em, text centered, minimum height=3em]
\tikzstyle{line} = [draw, -latex']

\begin{center}
\resizebox{0.8\linewidth}{!}{%
  \begin{minipage}{\linewidth}
    \centering
      \begin{tikzpicture}[->,>=stealth', auto, node distance=1cm, main node/.style={circle,fill=blue!15,draw,font=\sffamily\bfseries}]
      % place nodes
      \node [block, text width=13em] (1) {Bilddaten entstören};
      \node [block, below of = 1, node distance=1.5cm, text width=13em] (2) {Farbsegmentierung};
      \node [block, below of = 2, node distance=1.5cm, text width=13em] (3) {Farbsegmentierung invertieren};
      
      \path [line] (1) -- (2);
      \path [line] (2) -- (3);
      
      \end{tikzpicture}
  \end{minipage}
}
\end{center}
~ \\
~ \\
Farbsegmentierung für rot und blau \\
Beschränkung des Schildkataloges auf eine Schildauswahl
}


\subsection {Schilderkennung durch die Form}
\frame {\frametitle{\thesection.\thesubsection~ \insertsubsection}
% define block styles
\tikzstyle{block} = [rectangle,fill=gray!10,draw,text width=10em,text centered,minimum height=3em]
\tikzstyle{cloud} = [ellipse, fill=gray!10,draw, text width=5em, text centered, minimum height=3em]
\tikzstyle{line} = [draw, -latex']

\begin{center}
\resizebox{0.8\linewidth}{!}{%
  \begin{minipage}{\linewidth}
    \centering
      \begin{tikzpicture}[->,>=stealth', auto, node distance=1cm, main node/.style={circle,fill=blue!15,draw,font=\sffamily\bfseries}]
      % place nodes
      \node [block, text width=13em] (1) {Bild segmentieren};
      \node [block, below of = 1, node distance=1.5cm, text width=13em] (2) {Segmentierte Flächen\\ auf Formen überprüfen};
      
      \path [line] (1) -- (2);
      
      \end{tikzpicture}
  \end{minipage}
}
\end{center}
~ \\
~ \\
Zur Überprüfung sollen die Formeln aus dem Paper verwendet werden.
}


\subsection {Schildklassifikation}
\frame {\frametitle{\thesection.\thesubsection~ \insertsubsection}
Schildklassifikation durch:\\
\begin{itemize}
 \item Form\\
 \item Farbe\\
 \end{itemize}
 ~ \\
 In die Klassen: \\
 ~ \\
\begin{tabular}{l l l}
\hline
\textbf{Farbe} & \textbf{Form} & \textbf{Kategorie} \\
\hline
Rot & Oktagonal & Stop \\ 
Rot & Dreieck, Spitze unten & Vorfahrt gewähren \\ 
Rot & Dreieck, Spitze oben & Warnung \\ 
Rot & Kreis & Verbot \\ 
Blau & Kreis & Verpflichtend \\
\end{tabular}
}


\subsection {Schild entzerren \& erkennen}
\frame {\frametitle{\thesection.\thesubsection~ \insertsubsection}
 Diesen Punkt haben wir optional angedacht. \\
~ \\
~ \\
\begin{itemize}
 \item erkannter Bereich wird entzerrt und auf eine einheitliche Größe gebracht\\
 \item Vergleich mit Bibliotheksdaten\\
\end{itemize}
}

\end{document}

